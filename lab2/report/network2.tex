\documentclass[a4paper,12pt]{article}

\usepackage[hidelinks]{hyperref}
\usepackage{amsmath}
\usepackage{mathtools}
\usepackage{shorttoc}
\usepackage{cmap}
\usepackage[T2A]{fontenc}
\usepackage[utf8]{inputenc}
\usepackage[english, russian]{babel}
\usepackage{xcolor}
\usepackage{graphicx}
\usepackage{float}
\graphicspath{{./img/}}

\definecolor{linkcolor}{HTML}{000000}
\definecolor{urlcolor}{HTML}{0085FF}
\hypersetup{pdfstartview=FitH,
            linkcolor=linkcolor,
            urlcolor=urlcolor,
            colorlinks=true}

\DeclarePairedDelimiter{\floor}{\lfloor}{\rfloor}

\renewcommand*\contentsname{Содержание}

\newcommand{\plot}[3]{
    \begin{figure}[H]
        \begin{center}
            \includegraphics[scale=0.6]{#1}
            \caption{#2}
            \label{#3}
        \end{center}
    \end{figure}
}

\begin{document}
    \begin{titlepage}
	\begin{center}
		{\large Санкт-Петербургский политехнический университет\\Петра Великого\\}
	\end{center}
	
	\begin{center}
		{\large Физико-механический иститут}
	\end{center}
	
	
	\begin{center}
		{\large Кафедра «Прикладная математика»}
	\end{center}
	
	\vspace{8em}
	
	\begin{center}
		{\bfseries Отчёт по лабораторной работе №1 \\по дисциплине «Компьютерные сети»}
	\end{center}
	
	\vspace{5em}
	
	\begin{flushleft}
		\hspace{16em}Выполнил студент:\\\hspace{16em}Бочкарев Илья Алексеевич\\\hspace{16em}группа: 5040102/20201
		
		\vspace{2em}
		
		\hspace{16em}Проверил:\\\hspace{16em}к.ф.-м.н., доцент\\\hspace{16em}Баженов Александр Николаевич
		
	\end{flushleft}
	
	
	\vspace{6em}
	
	
	\begin{center}
		Санкт-Петербург\\2023 г.
	\end{center}	
	
\end{titlepage}
    \newpage

    \tableofcontents
    \listoffigures
    \listoftables
    \newpage

    \section{Постановка задачи}
    \quad Требуется реализовать протокол маршрутизации OSPF (Open Shortest Path First).
    Проверить его работоспособность на следующих видах топологий сети:
    линейная, кольцевидная и звёздная.

    \section{Теория}
    \quad OSPF (Open Shortest Path First) — протокол динамической маршрутизации,
    основанный на технологии отслеживания состояния канала
    и использующий для нахождения кратчайшего пути алгоритм Дейкстры.

    Описание работы протокола.
    \begin{itemize}
        \item После включения маршрутизаторов протокол ищет непосредственно подключенных соседей
        и устанавливает с ними «дружеские» отношения.
        \item Затем они обмениваются друг с другом информацией о подключенных и доступных им сетях.
        То есть они строят карту сети (топологию сети).
        Данная карта одинакова на всех маршрутизаторах.
        \item На основе полученной информации запускается алгоритм SPF (Shortest Path First),
        который рассчитывает оптимальный маршрут к каждой сети.
        Данный процесс похож на построение дерева, корнем которого является сам маршрутизатор,
        а ветвями — пути к доступным сетям.
    \end{itemize}

    \section{Реализация}
    \quad Весь код написан на языке Python (версии 3.9.5).
    \href{https://github.com/BoIlAl/Networks/tree/master/lab2}{Ссылка на GitHub с исходным кодом}.

    \section{Результаты}
    \quad Количество узлов во всех топологиях равно 5. Рассмотриваем линейную топологию с радиусом соединения 5, кольцевую топологию с радиусом соединения 6, звездную топологию с радиусом соединения 5.

    Для сети с линейной топологией имитируем падение одного из некрайних узлов и перенумируем оставшиеся узлы сети. Для сети с кольцевой топологией Имитируем падение случайного узла и перенумируем оставшиеся узлы сети. Для сети со звездной топологией имитируем падение центрального узла.

    Результаты приведены ниже.

    \plot{img/TopologyLine.png}{Сеть с линейной топологией}{p:tl}

    \begin{table}[H]
    \caption{Список путей в сети с линейной топологией}
        \begin{tabular}{| c | c | c | c | c | c |}
            \hline
             & 0 & 1 & 2 & 3 & 4 \\
            \hline
            0 & 0 & 0-1 & 0-1-2 & 0-1-2-3 & 0-1-2-3-4 \\
            \hline
            1 & 1-0 & 1 & 1-2 & 1-2-3 & 1-2-3-4 \\
            \hline
            2 & 2-1-0 & 2-1 & 2 & 2-3 & 2-3-4 \\
            \hline
            3 & 3-2-1-0 & 3-2-1 & 3-2 & 3 & 3-4 \\
            \hline
            4 & 4-3-2-1-0 & 4-3-2-1 & 4-3-2 & 4-3 & 4\\
            \hline
        \end{tabular}
        \centering
    \end{table}
    
    \plot{img/TopologyLine corrupted.png}{Поврежденная сеть с линейной топологией}{p:tl_c}

    \begin{table}[H]
    \caption{Список путей в поврежденнной сети с линейной топологией}
        \begin{tabular}{| c | c | c | c | c |}
            \hline
             & 0 & 1 & 2 & 3 \\
            \hline
            0 & 0 & 0-1 & - & - \\
            \hline
            1 & 1-0 & 1 & - & - \\
            \hline
            2 & - & - & 2 & 2-3 \\
            \hline
            3 & - & - & 3-2 & 3 \\
            \hline
        \end{tabular}
        \centering
    \end{table}

    \plot{img/TopologyRing.png}{Сеть с кольцевой топологией}{p:rl}

    \begin{table}[H]
    \caption{Список путей в сети с кольцевой топологией}
        \begin{tabular}{| c | c | c | c | c | c |}
            \hline
             & 0 & 1 & 2 & 3 & 4 \\
            \hline
            0 & 0 & 0-1 & 0-1-2 & 0-4-3 & 0-4 \\
            \hline
            1 & 1-0 & 1 & 1-2 & 1-2-3 & 1-0-4 \\
            \hline
            2 & 2-1-0 & 2-1 & 2 & 2-3 & 2-3-4 \\
            \hline
            3 & 3-4-0 & 3-2-1 & 3-2 & 3 & 3-4 \\
            \hline
            4 & 4-0 & 4-0-1 & 4-3-2 & 4-3 & 4 \\
            \hline
        \end{tabular}
        \centering
    \end{table}

    \plot{img/TopologyRing corrupted.png}{Поврежденная сеть с кольцевой топологией}{p:rl_c}

    \begin{table}[H]
    \caption{Список путей в поврежденнной сети с кольцевой топологией}
        \begin{tabular}{| c | c | c | c | c |}
            \hline
             & 0 & 1 & 2 & 3 \\
            \hline
            0 & 0 & 0-1 & 0-3-2 & 0-3 \\
            \hline
            1 & 1-0 & 1 & 1-0-3-2 & 1-0-3 \\
            \hline
            2 & 2-3-0 & 2-3-0-1 & 2 & 2-3 \\
            \hline
            3 & 3-0 & 3-0-1 & 3-2 & 3 \\
            \hline
        \end{tabular}
        \centering
    \end{table}

    \plot{img/TopologyStar.png}{Сеть со звездной топологией}{p:sl}

    \begin{table}[H]
    \caption{Список путей в сети со звездной топологией}
        \begin{tabular}{| c | c | c | c | c | c |}
            \hline
             & 0 & 1 & 2 & 3 & 4 \\
            \hline
            0 & 0 & 0-4-1 & 0-4-2 & 0-4-3 & 0-4 \\
            \hline
            1 & 1-4-0 & 1 & 1-4-2 & 1-4-3 & 1-4 \\
            \hline
            2 & 2-4-0 & 2-4-1 & 2 & 2-4-3 & 2-4 \\
            \hline
            3 & 3-4-0 & 3-4-1 & 3-4-2 & 3 & 3-4 \\
            \hline
            4 & 4-0 & 4-1 & 4-2 & 4-3 & 4 \\
            \hline
        \end{tabular}
        \centering
    \end{table}
    
    \plot{img/TopologyStar corrupted.png}{Поврежденная сеть со звездной топологией}{p:sl_c}

    \begin{table}[H]
    \caption{Список путей в поврежденнной сети со звездной топологией}
        \begin{tabular}{| c | c | c | c | c |}
            \hline
             & 0 & 1 & 2 & 3 \\
            \hline
            0 & 0 & - & - & - \\
            \hline
            1 & - & 1 & - & - \\
            \hline
            2 & - & - & 2 & - \\
            \hline
            3 & - & - & - & 3 \\
            \hline
        \end{tabular}
        \centering
    \end{table}

    \section{Обсуждение}
    \quad Из полученных результатов можно заметить следующее.
    Сеть с линейной топологией крайне чувствительна к потерям
    некраевых узлов сети, потеря одного такого узла ведёт к появлению недостижимых узлов. Кроме того, максимальная длина пути равна $ n - 1 $, где $n$ -- количество узлов.
    Сеть с кольцевидной топологией менее чувствительна к потерям узлов,
    при потере любого узла она становится сетью с линейной топологией. Кроме того, максимальная длина пути равна $ \lfloor n / 2 \rfloor $, где $n$ -- количество узлов.
    Сеть со звёздной топологией наименее чувствительна к потере узлов до тех пор, пока это не центральный узел. В случае потери центрального узла любая пара других узлов становится недостижима. Кроме того, максимальная длина пути в равна $ 2 $.  

\end{document}