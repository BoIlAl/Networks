\documentclass[a4paper,12pt]{article}

\usepackage[hidelinks]{hyperref}
\usepackage{amsmath}
\usepackage{mathtools}
\usepackage{shorttoc}
\usepackage{cmap}
\usepackage[T2A]{fontenc}
\usepackage[utf8]{inputenc}
\usepackage[english, russian]{babel}
\usepackage{xcolor}
\usepackage{graphicx}
\usepackage{float}
\graphicspath{{./img/}}

\definecolor{linkcolor}{HTML}{000000}
\definecolor{urlcolor}{HTML}{0085FF}
\hypersetup{pdfstartview=FitH,
            linkcolor=linkcolor,
            urlcolor=urlcolor,
            colorlinks=true}

\DeclarePairedDelimiter{\floor}{\lfloor}{\rfloor}

\renewcommand*\contentsname{Содержание}

\newcommand{\plot}[3]{
    \begin{figure}[H]
        \begin{center}
            \includegraphics[scale=0.6]{#1}
            \caption{#2}
            \label{#3}
        \end{center}
    \end{figure}
}

\begin{document}
    \include{title}
    \newpage

    \tableofcontents
    \listoffigures
    \newpage

    \section{Постановка задачи}
    \quad Нужно реализовать систему Sender-Receiver,
    в которой участники будут обмениваться сообщениями по каналу связи с помощью протоколов автоматического запроса повторной передачи
    со скользящего окном: Go-Back-N и Selective Repeat.

    Необходимо выяснить зависимость времени работы и количество посланных сообщений от размера плавающего окна и вероятности потери сообщения для каждого протокола и сравнить друг с другом.

    \section{Теория}
    \quad Протоколы Go-Back-N и Selective Repeat являются протоколами скользящего окна: доставка сообщений происходит в рамках некоторого окна фиксированного размера. Ошибки выявляются и исправляются в рамках окна.

    Основное различие между этими двумя протоколами заключается в том, что после обнаружения подозрительного или поврежденного сообщения протокол Go-Back-N повторно передает все сообщения, не получившие подтверждения о получении, тогда как протокол Selective Repeat повторно передает только то сообщение, которое оказалось повреждено.

    \section{Реализация}
    \quad Весь код написан на языке Python (версии 3.9.5).
    \href{https://github.com/BoIlAl/Networks/tree/master/lab1}{Ссылка на GitHub с исходным кодом}.

    \section{Результаты}
    \quad Cравниваются  оба протокола по числу сообщений, которые пришлось отправить, а также по времени
    работы, необходимому для получения всех сообщений без ошибок.
    
    Рассмотрим зависимость этих метрик от размера окна и вероятности потери сообщения.

    По умолчанию число сообщений равно 100, ширина окна 15, вероятность потери сообщения 0.3.

    \plot{img/GBN SR num ws.png}{Зависимость числа сообщений от ширины окна}{p:n_ws}

    \plot{img/GBN SR time ws.png}{Зависимость времени работы от ширины окна}{p:t_ws}

    \plot{img/GBN SR num prob.png}{Зависимость числа сообщений от вероятности потери сообщения}{p:n_pr}

    \plot{img/GBN SR time prob.png}{Зависимость времени работы от вероятности потери сообщения}{p:t_pr}

    \section{Обсуждение}
    \quad Из приведённых результатов можно заметить, что в одинаковых условиях протоколу Selective Repeat требуется отправить меньше сообщений, чем протоколу Go-Back-N.
    Что ожидаемо, в силу разной обработки и повторной передачи потерянных сообщений.
    Протокол Selective Repeat работает значительно быстрее протокола Go-Back-N.

\end{document}