\documentclass[a4paper,12pt]{article}

\usepackage[hidelinks]{hyperref}
\usepackage{amsmath}
\usepackage{mathtools}
\usepackage{shorttoc}
\usepackage{cmap}
\usepackage[T2A]{fontenc}
\usepackage[utf8]{inputenc}
\usepackage[english, russian]{babel}
\usepackage{xcolor}
\usepackage{graphicx}
\usepackage{float}
\graphicspath{{./img/}}

\definecolor{linkcolor}{HTML}{000000}
\definecolor{urlcolor}{HTML}{0085FF}
\hypersetup{pdfstartview=FitH,
            linkcolor=linkcolor,
            urlcolor=urlcolor,
            colorlinks=true}

\DeclarePairedDelimiter{\floor}{\lfloor}{\rfloor}

\renewcommand*\contentsname{Содержание}

\newcommand{\plot}[3]{
    \begin{figure}[H]
        \begin{center}
            \includegraphics[scale=0.6]{#1}
            \caption{#2}
            \label{#3}
        \end{center}
    \end{figure}
}

\begin{document}
    \begin{titlepage}
	\begin{center}
		{\large Санкт-Петербургский политехнический университет\\Петра Великого\\}
	\end{center}
	
	\begin{center}
		{\large Физико-механический иститут}
	\end{center}
	
	
	\begin{center}
		{\large Кафедра «Прикладная математика»}
	\end{center}
	
	\vspace{8em}
	
	\begin{center}
		{\bfseries Отчёт по лабораторной работе №1 \\по дисциплине «Компьютерные сети»}
	\end{center}
	
	\vspace{5em}
	
	\begin{flushleft}
		\hspace{16em}Выполнил студент:\\\hspace{16em}Бочкарев Илья Алексеевич\\\hspace{16em}группа: 5040102/20201
		
		\vspace{2em}
		
		\hspace{16em}Проверил:\\\hspace{16em}к.ф.-м.н., доцент\\\hspace{16em}Баженов Александр Николаевич
		
	\end{flushleft}
	
	
	\vspace{6em}
	
	
	\begin{center}
		Санкт-Петербург\\2023 г.
	\end{center}	
	
\end{titlepage}
    \newpage

    \tableofcontents
    \listoffigures
    \listoftables
    \newpage

    \section{Постановка задачи}
    \quad Требуется реализовать решение задачи византийских генералов алгоритмом Лампорта.

    \section{Теория}
    \quad Имеется $ n $ генералов, из которых $ m $ предатели.
    Между каждым из $ n $ генералов установлен надёжный (исключающий подмену сообщения) канал связи.
    Каждый из $ n - m $ верных генералов каждый раз посылает истинное и неизменяемое сообщение,
    а каждый из $ m $ предателей посылает ложное и, возможно, изменяемое сообщение.
    Верным генералам, в результате обмена сообщений, необходимо определить предателей.

    Будем решать задачу в частном случае, когда число предателей не меняется.
    Для этого случая существует алгоритм Лампорта, который состоит из следующих шагов.

    \begin{itemize}
        \item Каждый генерал посылает всем остальным сообщение, верные генералы - истинное, предатели - ложное.
        \item В результате у каждого генерала формируется массив из $ n $ элементов (полученных сообщений, включая и своё)
        \item Каждый генерал посылает всем остальным полученный на прошлом шаге массив.
        \item В конце каждый генерал имеет набор векторов, свой и полученный от других генералов.
        Для каждого $ i $ элемента каждого вектора находится то, которое чаще других встречается.
        Если оно встречается как минимум $ n - m $ раз, то оно считается истинным и помещается в результирующий вектор,
        иначе в результирующий вектор помещается нуль. 
    \end{itemize}

    Доказано, что генералы всегда придут к согласию при условии $ 3m < n $.

    \section{Реализация}
    \quad Весь код написан на языке Python (версии 3.9.5).
    \href{https://github.com/BoIlAl/Networks/tree/master/lab3}{Ссылка на GitHub с исходным кодом}.

    \section{Результаты}
    \quad Рассмотрим пример работы алгоритма на модельном случае с $n = 7, \; m = 2$. В качестве индексов сопоставим генералам числа от 0 до 6 включительно. Последний генерал будет византийским, остальные -- честными.  Византийский генерал будет на первом этапе отправлять значения вида $v\_i$, где $i$ -- индекс генерала, которому адресовано сообщение, а на втором шаге -- $v\_i\_v\_j$, где $i$ -- индекс генерала, которому адресовано сообщение, $j$ -- индекс генерала, от которого (как утверждает византийский генерал) было получено это значение на первом этапе.

    Итоговые вектора (прочерк на месте генералов-предателей):

    \begin{equation}
        \begin{gathered}
            0 : [0, 1, 2, 3, 4, 5, 6, -, -], \\
            1 : [0, 1, 2, 3, 4, 5, 6, -, -],  \\
            2 : [0, 1, 2, 3, 4, 5, 6, -, -], \\
            3 : [0, 1, 2, 3, 4, 5, 6, -, -], \\
            4 : [0, 1, 2, 3, 4, 5, 6, -, -], \\
            5 : [0, 1, 2, 3, 4, 5, 6, -, -], \\
            6 : [0, 1, 2, 3, 4, 5, 6, -, -], \\
        \end{gathered}
    \end{equation}

    Генералы пришли к согласию. 
    В качестве альтернативы рассмотрим модельный случай с $n = 4, \; m = 3$.
    \begin{equation}
        \begin{gathered}
            0 : [v\_0\_0, v\_0\_v\_3, v\_0\_v\_3, v\_0\_v\_1], \\
            1 : [v\_1\_0, v\_1\_v\_3, v\_1\_v\_3, v\_1\_v\_2],  \\
            2 : [v\_2\_0, v\_2, v\_2\_v\_3, v\_2], \\
            3 : [v\_3\_0, v\_3\_v\_2, v\_3, v\_3\_v\_2] \\
        \end{gathered}       
    \end{equation}

    Условие $ 3m < n $ не выполнилось, генералы не пришли к согласию. Построим также таблицу успешности результатов для $ n = \overline{2, 7} $, $ m = \overline{0, n - 1} $

    \begin{table}[H]
    \caption{результаты решения задачи при разных $n, m$}
        \begin{tabular}{| c | c | c | c | c | c | c | c |}
            \hline
             n | m & 0 & 1 & 2 & 3 & 4 & 5 & 6 \\
            \hline
            2 & + & - & & & & &  \\
            \hline
            3 & + & + & - & & & & \\
            \hline
            4 & + & + & - & - & & & \\
            \hline
            5 & + & + & - & - & - & & \\
            \hline
            6 & + & + & + & - & - & - & \\
            \hline
            7 & + & + & + & - & - & - & - \\
            \hline
        \end{tabular}
        \centering
    \end{table}

    \section{Обсуждение}
    \quad В результате работы реализован алгоритм Лампорта для решения частного случая задачи Византийских генералов. Показана работоспособность алгоритма. Кроме того, получен результат, говорящий о том, что условие $ 3m < n $ является досаточным, но не необходимым для успешного решения.

\end{document}